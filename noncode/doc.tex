
\documentclass[12pt]{article}
\usepackage{listings}
\usepackage{pdfpages}
\usepackage{amsmath}
\usepackage[utf8]{inputenc}
\usepackage[english]{babel}
\usepackage{multicol}
\usepackage{babel}
\usepackage{graphicx}
%\usepackage{tgbonum}
\usepackage{mathptmx}

\usepackage[margin=1in]{geometry}
\begin{document}
\begin{titlepage}
   \begin{center}
       \vspace*{1cm}
       \Large
       Project 4: RSA, DFT, FFT
       \normalsize

       \vspace{0.5cm}

       Author: Gabriel Hofer

       \vspace{0.5cm}

       Course: CSC-372 Analysis of Algorithms

       \vspace{0.5cm}

       Instructor: Dr. Rebenitsch

       \vspace{0.5cm}

       Due: November 19, 2020

       \vfill

       Computer Science and Engineering\

       South Dakota School of Mines and Technology\
   \end{center}
\end{titlepage}
\newpage
%------------------------------------------------------------------------------------
%------------------------------------------------------------------------------------
\small
\newpage
\subsection*{Question 1}

[20 points] use the RSA algorithm. If $p=13$ and $q=15$, $e=11$  

\noindent \textbf{[2 points] What is n?}  \\
\[
  n = p \cdot q 
\]
\[
  n = 195
\]

\noindent \textbf{[2 points] What is $\varphi(n)$?} \\ 

$\varphi $ is Euler's totient function.
\[
  \varphi(n) = 96
\]

\noindent \textbf{[2 points] Name an invalid $e$ for this problem.} \\

$\lambda$ is Carmichael's function. 
\[
  \lambda(n) = 12
\]
\noindent $e$ must be in the range $ 1 < e < \lambda(n)$ and $ gcd(e,\lambda(n)) = 1 $.
So, an example of an invalid $e$ would be 4 since 
\[
  gcd(4,\lambda(195)) = 4
\]

\noindent \textbf{[6 points] What is d (you MUST show your work for credit)?} \\

\noindent We want to find the modulo multiplicative-inverse of $e$ modulo $\lambda(n)$. 
Fermat's Little Theorem:
\[
  e^{\lambda(n)-1} \ \equiv \ \ 1 \ (mod \ \lambda(n))
\]
Multiply both sides by $a^{-1}$.
\[
  e^{\lambda(n)-2} \ \equiv \ \ e^{-1} \ (mod \ \lambda(n))
\]
Simplify.
\[
  1 \ \equiv \ \ e^{-1} \ (mod \ \lambda(n))
\]
So $d = 1$. \\

\noindent \textbf{[8 points] Use the above values to encode 5 with $e$ (use the MOD-Exp funtion
and show the values for each iteration). You should only [show] the first 5 
iterations rather than all $e$ interations.}

\newpage
\subsection*{Question 3}
\noindent \textbf{[30 points] Compute the DFT for n=6 and $f(x) = 3x^5 + 4x^4 - 2x^3 - x^2 + 4$,
for the $2^{nd}$ power $(w_6^2)$ Note the missing powers! It must be clear that this
is the DFT (so a tree-like structure would be best). You must show your work for credit. 
Your answers must be in $a + bi$ format.}

\[
  f(x) = 3x^5 + 4x^4 - 2x^3 - x^2 + 4 \ \ \Rightarrow \ \ \textbf{x} = 
  \begin{pmatrix}
    x_0 \\ x_1 \\ x_2 \\ x_3 \\ x_4 \\ x_5
  \end{pmatrix} 
  = 
\begin{pmatrix}
  4 \\ 0 \\ -1 \\ -2 \\ 4 \\ 3
\end{pmatrix}
\]


\[
  X_0 = e^{-i2\pi 0 \cdot 0/6} \cdot 4 + e^{-i2\pi 0 \cdot 1/6} \cdot 0 + e^{-i2\pi 0 \cdot 2/6} \cdot -1 +  e^{-i2\pi 0 \cdot 3/6} \cdot -2 + e^{-i2\pi 0 \cdot 4/6} \cdot 4 + e^{-i2\pi 0 \cdot 5/6} \cdot 3
\]                                                                                                                                                                                                            
\[                                                                                                                                                                                                            
  X_1 = e^{-i2\pi 1 \cdot 0/6} \cdot 4 + e^{-i2\pi 1 \cdot 1/6} \cdot 0 + e^{-i2\pi 1 \cdot 2/6} \cdot -1 +  e^{-i2\pi 1 \cdot 3/6} \cdot -2 + e^{-i2\pi 1 \cdot 4/6} \cdot 4 + e^{-i2\pi 1 \cdot 5/6} \cdot 3
\]                                                                                                                                                                                                            
\[                                                                                                                                                                                                            
  X_2 = e^{-i2\pi 2 \cdot 0/6} \cdot 4 + e^{-i2\pi 2 \cdot 1/6} \cdot 0 + e^{-i2\pi 2 \cdot 2/6} \cdot -1 +  e^{-i2\pi 2 \cdot 3/6} \cdot -2 + e^{-i2\pi 2 \cdot 4/6} \cdot 4 + e^{-i2\pi 2 \cdot 5/6} \cdot 3
\]                                                                                                                                                                                                            
\[                                                                                                                                                                                                            
  X_3 = e^{-i2\pi 3 \cdot 0/6} \cdot 4 + e^{-i2\pi 3 \cdot 1/6} \cdot 0 + e^{-i2\pi 3 \cdot 2/6} \cdot -1 +  e^{-i2\pi 3 \cdot 3/6} \cdot -2 + e^{-i2\pi 3 \cdot 4/6} \cdot 4 + e^{-i2\pi 3 \cdot 5/6} \cdot 3
\]                                                                                                                                                                                                            
\[                                                                                                                                                                                                            
  X_4 = e^{-i2\pi 4 \cdot 0/6} \cdot 4 + e^{-i2\pi 4 \cdot 1/6} \cdot 0 + e^{-i2\pi 4 \cdot 2/6} \cdot -1 +  e^{-i2\pi 4 \cdot 3/6} \cdot -2 + e^{-i2\pi 4 \cdot 4/6} \cdot 4 + e^{-i2\pi 4 \cdot 5/6} \cdot 3
\]                                                                                                                                                                                                            
\[                                                                                                                                                                                                            
  X_5 = e^{-i2\pi 5 \cdot 0/6} \cdot 4 + e^{-i2\pi 5 \cdot 1/6} \cdot 0 + e^{-i2\pi 5 \cdot 2/6} \cdot -1 +  e^{-i2\pi 5 \cdot 3/6} \cdot -2 + e^{-i2\pi 5 \cdot 4/6} \cdot 4 + e^{-i2\pi 5 \cdot 5/6} \cdot 3
\]
 
 \[
   X = \begin{pmatrix}
     X_0 \\ X_1 \\ X_2 \\ X_3 \\ X_4 \\ X_5
   \end{pmatrix}
  =
  \begin{pmatrix}
    8+0i \\ 6+6.928i \\ -1-1.732i \\ 6+0i \\ -1+1.732i \\ 6-6.928i
  \end{pmatrix}
\]





\end{document}





